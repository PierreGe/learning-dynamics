              
% --------------------------------------------------------------
% This is all preamble stuff that you don't have to worry about.
% Head down to where it says "Start here"
% --------------------------------------------------------------
 
\documentclass[11pt]{article}
\usepackage[utf8]{inputenc} 
\usepackage[margin=1in]{geometry}
\usepackage{graphicx}
\usepackage{float}
\usepackage{hyperref} 
\usepackage{amsmath,amsthm,amssymb}
\usepackage[table]{xcolor}
 
\newcommand{\N}{\mathbb{N}}
\newcommand{\Z}{\mathbb{Z}}
 
\newenvironment{theorem}[2][Theorem]{\begin{trivlist}
\item[\hskip \labelsep {\bfseries #1}\hskip \labelsep {\bfseries #2.}]}{\end{trivlist}}
\newenvironment{lemma}[2][Lemma]{\begin{trivlist}
\item[\hskip \labelsep {\bfseries #1}\hskip \labelsep {\bfseries #2.}]}{\end{trivlist}}
\newenvironment{exercise}[2][Exercise]{\begin{trivlist}
\item[\hskip \labelsep {\bfseries #1}\hskip \labelsep {\bfseries #2.}]}{\end{trivlist}}
\newenvironment{reflection}[2][Reflection]{\begin{trivlist}
\item[\hskip \labelsep {\bfseries #1}\hskip \labelsep {\bfseries #2.}]}{\end{trivlist}}
\newenvironment{proposition}[2][Proposition]{\begin{trivlist}
\item[\hskip \labelsep {\bfseries #1}\hskip \labelsep {\bfseries #2.}]}{\end{trivlist}}
\newenvironment{corollary}[2][Corollary]{\begin{trivlist}
\item[\hskip \labelsep {\bfseries #1}\hskip \labelsep {\bfseries #2.}]}{\end{trivlist}}
 
\begin{document}
 
% --------------------------------------------------------------
%                         Start here
% --------------------------------------------------------------
 
\setlength\parindent{0pt}
%\renewcommand{\qedsymbol}{\filledbox}
 
\title{Assignment 1: Game theory}%replace X with the appropriate number
\author{Pierre Gérard (ULB)\\ %replace with your name
INFO-F-409 - Learning dynamics} %if necessary, replace with your course title
 
\maketitle

\begin{exercise}{1} The Hawk-Dove game
\end{exercise}

Blablabla

\begin{exercise}{2} Which social dilemma?
\end{exercise}

For player A, each social dilemma has an equal probability of $\frac{1}{3}$ of being played. Player A does not know and which game will be play and has to choose a strategy in the set $\{C, D \}$ . 

On the other hand, Player B knows in advance in which game he will be playing and had to choose a strategy in $\{C, D \}$ for every game, resulting in strategy in the set $\{ CCC, CDC, CCD, ... \} $

Let's modelize this uncertain environment to find the best strategy for each player in response to the other one.

\subsubsection{Player A}

To find the best response of player A to a strategy of player B, let's find the payoff of player A for every strategy of Player B. For that we will give the details for some computation and a table for all the result. Like in the slides, red means best strategy.


\subsubsection{Player B}

To find the best response of player B to a strategy of player A, let's find the payoff of player B for every strategy of Player A.



The response that will be best for Player B if A choose $C$ is ${CCD}$ and $DDC$ if player A choose $D$. 


\textbf{When matching the result we find that there is two Nash equilibras :
$(C - DCD)$ and $(D - DDC)$ }

\begin{exercise}{3} Sequential truel
\end{exercise}

Blablabla
 

 
% --------------------------------------------------------------
%     You don't have to mess with anything below this line.
% --------------------------------------------------------------
 
\end{document}







































              